% $Id: Alarm_usage.tex,v 1.6 2005/06/23 04:38:13 eschwab Exp $

Alarms are used in conjunction with Clocks (see Section~\ref{sec:Clock}).
Multiple Alarms can be associated with a Clock.  During the
{\tt ESMF\_ClockAdvance()} method, a Clock iterates over its internal Alarms
to determine if any are ringing.  Alarms ring when a specified Alarm 
time is reached or exceeded, taking into account whether the time step is
positive or negative.  In {\tt ESMF\_MODE\_REVERSE}
(see Section~\ref{sec:Clock}), alarms ring in reverse, i.e., they begin
ringing when they originally ended, and end ringing when they originally
began.  On completion of the time advance call, the Clock optionally returns
a list of ringing alarms.

Each ringing Alarm can then be processed using Alarm methods for identifying,
turning off, disabling or resetting the Alarm.

Alarm methods are defined for obtaining the ringing state, turning the
ringer on/off, enabling/disabling the Alarm, and getting/setting 
associated times.

The following example shows how to set and process Alarms.
