% $Id: ESMF_shareobj.tex,v 1.5 2006/12/13 20:18:01 cdeluca Exp $

\subsection{Shared Objects}
 Using shared libraries you can: 
\begin{itemize}
\item update libraries and still support programs that want to use older, non-backward-compatible versions of those libraries;
\item override specific libraries or even specific functions in a library when executing a particular program.
\item do all this while programs are running using existing libraries.
\end{itemize}

Shared objects libraries have two different names: the \emph{soname} and the \emph{real name}. The \emph{soname} consists of the prefix \emph{"lib"}, followed by the name of the library, a \emph{".so"} followed by another dot, and a number indicating the major version number. The \emph{soname} may be fully qualified by prefixing path information. The \emph{real name} is the actual file name containing the compiled code for the library. The real name adds a dot, a minor number, another dot, and the release number, to the \emph{soname}. (The release number and the associated dot are optional.)

The use of shared object libraries:
\begin{itemize}
\item simplifies the process of compiling and linking mixed C/C++ and F90 source code on multiple compiler/platform combinations. 
\item allows library upgrades to be distributed and installed separately from the applications that use them. 
\item increases program modularity and can make it easier to manage the evolution of large applications.
\item shortens the time required to recompile applications.
\end{itemize}

\subsubsection{Can I mix shared object libraries with other types of libraries?}

In short yes. But, there are a few things to keep in mind when including them in your application:

Static vs Dynamic - If there are two copies of a library, one static and one shared, the default behavior is to prefer the shared library. Before looking for static library like \emph{'libutil.a'}, the linker will look for a file named \emph{'libutil.so'} - as a shared library. Only if it cannot find a shared library, will it look for \emph{'libutil.a'} as a static library. This behavior can be overridden using some linker flags (\emph{'-Wl,static'} with some linkers, \emph{'-Bstatic'} with other types of linkers. refer to the compiler's or the linker's manual for info about these flags). 

%[TODO:] Answer if this is a potential problem for users component code to end up loading duplicate copies of .so libraries.�(Suppose you are building two dynamic-load modules, B and C, which should share another block of code A. On Unix, you would not pass A.a to the linker for B.so and C.so; that would cause it to be included twice, so that B and C would each have their own copy.)


































